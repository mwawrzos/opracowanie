\chapter{Języki i~metody programowania}
\PartialToc
%\startcontents[chapters]
%\printcontents[chapters]{}{1}{\section*{\contentsname}}

\section{K\_W15}
\textbf{296. Dla domkniecia Kleene’ego prawdziwe sa nastepujace stwierdzenia:}\\
przykładowa odp.) jest to zbiór słów powstały przez konkatencje dwóch domknietych dodatnio jezyków\\
\textbf{FAŁSZ}

Domknięcie Kleene'ego L* języka L jest najmniejszym spośród zbiorów X spełniających\\
\begin{enumerate}
\item $\varepsilon \in X$
\item $L \subseteq X$
\item jeżeli $s \in X$ oraz $x \in L$, to $sx \in X $
\end{enumerate}

\textbf{Dodatnie} domknięcie Kleene'ego $L^+$ definiujemy jako najmnieszy spośród zbiorów X spełniających\\
\begin{enumerate}
\item $L \subseteq X$
\item jeżeli $s \in X$ oraz $x \in L$, to $sx \in X $
\end{enumerate}

Przykładowo:\\\\
$V = \{ba, ca\}$\\
$V* = \{\varepsilon, ba, ca, baba, baca, caca, caba, babaca, ...\}$\\
$V^+ = \{ba, ca, baba, baca, caca, caba, babaca, ...\}$




\section{K\_W15, K\_U15, K\_U21, K\_U22}
\textbf{300. Odnośnie lematu o pompowaniu dla języków regularnych prawdziwe są:}\\
przykładowa odp.) lemat słuzy pokazaniu, że określone języki są regularne\\
\textbf{FAŁSZ}\\

Twierdzenie służy do dowodzenia, że dany język \textbf{nie} jest językiem regularnym\\\\
Twierdzenie brzmi:\\
Jeżeli dany język $L$ jest regularny, to istnieje takie $n \geq 1$, że każde słowo w należące do $L$ długości co najmniej $n$ można podzielić na trzy części $xyv$, gdzie $y$ jest niepuste i $|xy| < n$, w taki sposób, że dla każdego $k \geq 0$ słowo postaci $xy^kv$ należy do danego języka.
\\



\section{K\_W15, K\_W03, K\_W04, K\_W07, K\_U15, K\_U21, K\_U22}
\textbf{308. Mamy języki $L_1 = \{ a^{2^n} : n > 0\}$ oraz $L_2 = \{ a^{2n} : n > 0\}$. Które z tych języków są regularne? }\\
przykładowa odp.) $L_1$ -- tak, $L_2$ -- tak\\
\textbf{FAŁSZ}\\\\

\textbf{Język regularny} -- język formalny taki, że istnieje automat o skończonej liczbie stanów potrafiący zdecydować, czy dane słowo należy do języka.\\\\

Wykorzystując lemat o pompowaniu należy pokażać, że język jest nieregularny. Dla $L_1$:\\
\begin{enumerate}
\item Wybieramy x,y,z takie, że : $x=\varepsilon, y=a^{2^{k}-1}, z=a$
\item Słowo składa się wyłącznie z liter $a$ oraz $0<|v|<2^k$
\item Dla $i=2$ słowo $xuv^2wz \notin L_1 $, bo $2^k < |xuv^2wz| < 2^{k+1}$
\end{enumerate}
Zatem $L_1$ jest językiem nieregularnym

