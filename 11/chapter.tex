\chapter{Sieci komputerowe}
\PartialToc
%\startcontents[chapters]
%\printcontents[chapters]{}{1}{\section*{\contentsname}}
\answer
{Adres typu broadcast (rozgłoszenia) IP w wersji 4 dla sieci IP, w której
znajduje się host 110.104.1.10 i która okresla maska 255.0.0.0, to}
{110.104.1.0}
{F} %należy wpisać "T" (prawda™) lub "F" (FAŁSZ) 
{110.255.255.255}
{Adresy broadcast dla IP wersji 4 to adres który w danej sieci bajtowo zapisujemy jako ciąg 1. 
Dla IP 110.104.1.10 o masce 255.0.0.0 adres sieci to 110.0.0.0 (wszystkie bajty w pierwszym członie adresu zostały zachowane (255 w masce). Oznacza to, iż w 3 kolejnych członach zapisanych bajtowo znajdą się same 1). Jako iż każdy człon składa się z 8 bajtów adres ten zapisany w systemie dziesiętnym będzie miał wartość 110.255.255.255 }

\answer
{Pole o nazwie Time to Live w datagramie IP, które zabezpiecza przed zapętleniem rutowania datagramu pomiędzy kolejnymi ruterami w sieci zawiera:}
{Liczbę ruterów przez jakie datagram IP może zostać przekazany dalej }
{T} %należy wpisać "T" (prawda™) lub "F" (FAŁSZ) 
{Liczbę ruterów przez jakie datagram IP może zostać przekazany dalej }
{W przypadku czasu życia pakietów danych w sieci komputerowej jest to zwykle liczba przeskoków, które może on wykonać na swojej trasie. Każdy kolejny router IP na trasie danego pakietu zmniejsza wartość jego pola TTL o jeden. Jeśli router otrzyma pakiet z TTL równym 1, zmniejsza go do 0, odrzuca go i usuwa z sieci, a nadawca otrzyma komunikat ICMP o błędzie. Czas życia pakietu pomaga unikać przeciążenia sieci w przypadku źle skonfigurowanych tras routingu w routerach, np. występowania pętli w sieci.}

\answer
{Nazwa ramki stosowanej w technologii IEEE 802.11 i emitowanej przez
urządzenie Access Point i stosowanej między innymi w celu propagowania informacji o sieci bezprzewodowej,
to}
{Checker}
{F} %należy wpisać "T" (prawda™) lub "F" (FAŁSZ) 
{Beacon frame }
{Beacon frame to jedna z tzw. Management frames w standardzie IEEE 802.11. Zawiera informacje o sieci. Ramki te wysyłąne są z Access Pointów co jakiś czas i rozgłaszają obecność sieci.}

\answer
{Protokół UDP definiuje identyfikatory przesyłanych do hosta-odbiorcy datagramów
zwane numerami portów, o długosci}
{32}
{T lub F (patrz wyjąśnienie)} %należy wpisać "T" (prawda™) lub "F" (FAŁSZ) 
{32 lub 16 }
{Nagłówek UDP rozpoczynają 2 pola Port nadawcy oraz port odbiorcy. Każdy po 16 bitów. Ich łączna długość to 32.

UDP jeden z protokołów internetowych. UDP stosowany jest w warstwie transportowej modelu OSI.

Jest to protokół bezpołączeniowy, więc nie ma narzutu na nawiązywanie połączenia i śledzenie sesji (w przeciwieństwie do TCP). Nie ma też mechanizmów kontroli przepływu i retransmisji. Korzyścią płynącą z takiego uproszczenia budowy jest większa szybkość transmisji danych i brak dodatkowych zadań, którymi musi zajmować się host posługujący się tym protokołem. Z tych względów UDP jest często używany w takich zastosowaniach jak wideokonferencje, strumienie dźwięku w Internecie i gry sieciowe, gdzie dane muszą być przesyłane możliwie szybko, a poprawianiem błędów zajmują się inne warstwy modelu OSI. Przykładem może być VoIP lub protokół DNS.
}

\answer
{Wartosci adresu IPv6 oraz maski, określające wszystkie hosty w Internecie, to}
{ ::/0}
{F} %należy wpisać "T" (prawda™) lub "F" (FAŁSZ) 
{2000::/3 ???}
{Prawdopodobnie chodzi tutaj o Global Unicast - w IPv4 jest to publiczny adres IP, czyli taki który jest rutowany w całym internecie.}


\answer
{Istnienie zasady “Longest prefix match“ w rutowaniu IP spowoduje, że adres docelowy 200.200.200.1 datagtramu IP przy istnieniu w tablicy rutowania jednoczesnie reguł o wzorcach i maskach (podano w notacji CIDR): 200.200.200.0/18, 200.200.200.0/20, 200.200.200.0/22, 200.200.200.0/24
zostanie dopasowany do:}
{200.200.200.0/20}
{F} %należy wpisać "T" (prawda™) lub "F" (FAŁSZ) 
{200.200.200.0/24}
{Longest Prefix Match to algorytm stosowany przez rutery przy wyborze kolejnego rutera z tablicy rutowania. Jako, iż wiele wpisów w owej tablicy może pasować do jednego adresu. (w powyższym przykładzie wszystkie 4 pasują do podanego) to wybierane jest wejście z najdłuższą pasującą maską podsieci (w tym wypadku o długości 24).}


\answer
{Maksymalna długość pakietu IP wersja 4, licząc w bajtach, to}
{1024}
{F} %należy wpisać "T" (prawda™) lub "F" (FAŁSZ) 
{65,535}
{Długość ta wynosi tyle ponieważ w nagłówku pakietu IP znajduje się pole TOTAL LENGTH (oznaczające długość całego pakietu w bajtach). Pole to zajmuje 2 bajty (16 bitów). Oznacza to iż maksymalna liczba jaką można tam zapisać to binarnie $ 2 ^ 16 -1 $ czyli 65535 }


\answer
{Okreslenie stosowane wobec rutera MPLS (MultiProtocol Label Switching), będącego w danej sytuacji odbiorcą datagramów z etykietami MPLS od innego (nie będącego przedmiotem rozwazań to:}
{ Designated router}
{F} %należy wpisać "T" (prawda™) lub "F" (FAŁSZ) 
{Label switch router ? a nie chodzi tu o Downstream Router czyli według wykładów - dla datagramów - ten który odbiera datagramy MPLS ????}
{MPLS to technika stosowana przez routery, w której trasowanie pakietów zostało zastąpione przez tzw. przełączanie etykiet.

Na brzegu sieci z protokołem MPLS do pakietu dołączana jest dodatkowa informacja zwana etykietą (ang. Label). Router po odebraniu pakietu z etykietą (jest to z punktu widzenia danego routera etykieta wejściowa) używa jej jako indeksu do wewnętrznej tablicy etykiet, w której znajduje się następny punkt sieciowy (ang. next hop) oraz nowa etykieta (etykieta wyjściowa). Etykieta wejściowa jest zastępowana wyjściową i pakiet jest wysyłany do następnego punktu sieciowego (np. do następnego routera). Jeżeli następny router nie obsługuje protokołu MPLS, etykieta jest usuwana.

Przypisanie pakietowi etykiety na brzegu sieci odbywa się w tzw. procesie klasyfikacji. Pakiety, które będą w jednakowy sposób routowane przez sieć MPLS, klasyfikowane są do jednej klasy FEC (ang. Forwarding Equivalence Class) i otrzymują tę samą etykietę. Przykładowo, klasy FEC mogą być budowane na bazie adresów docelowych IP w nagłówku pakietu w taki sposób, że każda klasa FEC pokrywa się z pojedynczym wpisem w tablicy trasowania routera.

Przyporządkowanie danej klasy FEC do etykiety jest sygnalizowane innym routerom za pomocą protokołu dystrybucji etykiet. Pozwala to routerom na zbudowanie tablic etykiet. W zależności od konkretnego zastosowania, jako protokół do dystrybucji etykiet używany jest LDP, albo odpowiednio rozszerzone protokoły: RSVP lub BGP.}

\answer
{Ruter iBGP (internal Border Gateway Protocol), którego wprowadzenie~do systemu rutowania iBGP umożliwia znaczne zredukowanie ilości otwartych sesji BGP pomiędzy innymi ruterami (rezygnację~z tzw. Full-mesh) nosi nazwę:}
{Route Reflector}
{T}
{Route Reflector}
{RR - Route Reflector - Dzieli obszary działania na części - radykalnie zmniejszając liczbę sesji pomiędzy ruterami iBGP. W jednym AS(Systemie autonomicznym) może występować wiele RR. Przekazuje od komunikaty od klientów RR do innych routerów iBGP.}

\answer
{Liczba klas CoS (Class~of Service), definiowanych przez podstawowy mechanizm implementacji QoS (Quality~of Service)~w Ethernet (czyli standard IEEE 802.1p),~to:}
{16}
{F}
{8}
{Class of Service - jest formą priorytetowego kolejkowania, która jest używana w protokołach sieciowych. W Ethernet przyjmuje wartości od 0 do 7.}

\answer
{Wariant protokołu STP (Spanning Tree Protocol, IEEE 802.1d) pozwalający~w technologii Ethernet~na logiczne grupowanie sieci VLAN (Virtual LAN)~i budowanie mniejszej liczby drzew rozpinających (po jednym Spanning Tree dla każdej zdefiniowanej grupy),!to:}
{PVSTP (Per VLAN Spanning Tree Protocol)}
{F}
{MSTP - Multiple Spanning Tree Protocol}
{Nie ma tu zbyt wiele do wyjaśniania}

\answer
{Rodzaje (grupy) urządzeń fizycznych definiowanych~w technologii ZigBee,~to:}
{ZigBee End Device, ZigBee Coordinator, ZigBee Router}
{T}
{ZigBee End Device, ZigBee Coordinator, ZigBee Router}
{ZED - ZigBee End Device - urządzenie czujnika, sterownika lub inne, które jest faktycznym źródłem lub odbiorcą danych.
ZC - ZigBee Coordinator - pełni rolę urządzenia zarządzającego parametrami transmisji, jest centralnym węzłem sieci (do niego podłączone są inne urząddzenia), może być mostkiem do innych sieci.
ZR - ZigBee Router - komunikuje inne urządzenia ZigBee, służy głównie do przedłużania zasięgu, lecz także komunikuje się~z innymi technologiami (network level).}

\answer
{Nazwa procesu przekazywania wiedzy o trasach pomiędzy różnymi protokołami rutowania dynamicznego IP w ruterach IP,~to:}
{Redystrybucja}
{T}
{Redystrybucja}
{Redystrybucja (protokołu) - zabieg przekazywania informacji o trasach pozyskanych z użyciem jednego  z obsługiwanych i działających protokołów rutowania.}

\answer
{Symbole literowe, określające rodzaje popularnych~w sieciach komputerowych wtyków światłowodowych,~to:}
{LC, SC, MTRJ}
{T}
{LC, SC, MTRJ}
{LC - najbardziej popularne złącze światłowodowe - simplex/duplex.
SC - poprawiona polaryzacja, większa stabilnośc mechaniczna łącza w porównaniu do simplex-ST.
MTRJ - ostatnio bardzo popularyzowane (tylko duplex)}

\answer
{Co określa standard IEEE 802.1Q?}
{Wirtualne sieci LAN (VLAN) budowane~w środowisku transportującym ramki}
{T}
{Wirtualne sieci LAN (VLAN) budowane~w środowisku transportującym ramki}
{Standard ten opisuje działanie wirtualnych sieci LAN (VLAN) budowanych w środowisku transportującym ramki}

\answer
{Protokół umożliwiający konwersję adresu IP zdalnej stacji~na jej adres MAC~w Ethernet,~to:}
{MLD (Multicast Listener Discovery)}
{F}
{ARP}
{ARP - Address Resolution Protocol -  protokół sieciowy umożliwiający mapowanie logicznych adresów warstwy sieciowej (warstwa 3 - IP) na fizyczne adresy warstwy łącza danych (2 - MAC).}

\answer
{Co zawiera pole Extended Unique Identifier (EUI) w adresie IPv6?}
{Adres IP w wersji 4 przypisany do stacji}
{F}
{Zawiera adres MAC urządzenia uzupełniony pośrodku o wartość: 0xFFFE}
{EUI - Extended Unique Identifier - adres MAC (48 bitów) zostaje uzupełniony pośrodku wartością 0xFFFE np. adres MAC - 00:0C:29:0C:47:D5, uzupełniony - 00:0C:29:FF:FE:0C:47:D5.}

\answer
{Domyślna wartość metryki Administrative Distance~w tablicy rutowania IP ruterów (np. Cisco, Juniper, Helwet Packard) przewidziana dla protokołu RIP (Routing Information Protocol),~to:}
{110}
{F}
{120}
{Tak zostało z góry ustalone.}

\answer
{W technologii Fibre Channel (stotowanej w sieciach SAN) port przełącznika Switch Fabric mogący pracować w topologii pętli arbitrażowej (pętli~z arbitrażem) sieci Fibre Channel, to port typu:}
{FL}
{T???????}
{FL}
{FL - Fabric-Loop Port, FL\_Port - port przełącznika mogący pracować~w topologii sieci arbitrażowej, używany do podłączenia portów NL do przełącznika.}

\answer
{Dwie pod-warstwy definiowane~w ramach warstwy drugiej modelu ISO-OSI to odpowiednio:}
{LAN i WAN}
{F}
{LLC i MAC}
{LLC - Logical Link Control, MAC - Media Access Control)}

\answer
{Zadaną~w jednostce dBm efektywną moc wypromieniowaną (Effective Isotropic Radiated Power, EIRP) bezprzewodowego urządzenia nadawczego stosowanego~w technologii sieciowej~na podstawie mocy wypromieniowanej P zadanej w watach można obliczyć stosujżc wzór:}
{EIRP = 1 / (P*1mW)}
{F}
{$EIRP = 10*log_{10}(\frac{P}{1mW})$}
{P - moc wypromieniowana}

\answer
{Jednostka wysokości urządzenia sieciowego montowanego~w standardzie RACK wynosząca 1,75 cala (44,45 mm) oznaczana jest symbolem:}
{h}
{F}
{U}
{U - najmniejsza możliwa wysokość urządzenia 1U = 1.75cala = 44.45mm}

\answer
{Rodzaj obszaru (area)~w domenie OSPF (Open Shortest Path First) nie otrzymującego żadnych informacji~o zewnętrznych (external) trasach rutowania OSPF,~to:}
{backbone}
{F}
{Stub area}
{Stub area - obszar wyizolowany - otrzymuje jedynie wyjście na zewnątrz, nie otrzymuje i nie wysyła informacji o sieciach.}

\answer
{Parametr~o nazwie "Wielkość okna"(Window size), którego wartość przekazywana jest~w datagramach potwierdzenia TCP (Transmission Control Protocol Acknowledgment)~w kierunku od odbiorcy do nadawcy ma na celu:}
{Określić ilość danych, jaką nadawca może~w danej chwili wysłać (służy do sterowania przepływem)}
{T?????? - w dobrą stronę ja myślę, bo już mi się miesza :/}
{Określić ilość danych, jaką nadawca może~w danej chwili wysłać (służy do sterowania przepływem)}
{Rozmiar okna - 16-bitowe pole używane przez komputer docelowy w celu poinformowania komputera źródłowego o tym, ile danych jest gotów przyjąć~w jednym segmencie TCP.}

\answer
{Dwa rodzaje obszarów (area)~w protokole rutowania dynamicznego IS-IS (Intermediate System to Intermediate System), to:}
{LAN i WAN}
{F}
{Level 1, Level 2}
{Level 1 - intra-area - traktowany jako backbone szkieletu sieci objętej protokołem, routery Level 1 komunikują się tylko z routerami Level 1.
Level 2 - inter-area - routery Level 2 komunikują się tylko z routerami Level 2.}
