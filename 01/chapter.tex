\chapter{Języki i~metody programowania}
\PartialToc
%\startcontents[chapters]
%\printcontents[chapters]{}{1}{\section*{\contentsname}}

% 1 ---------------------------------------------------------------------------------------------------------------

\section{IT1A\_W03,IT1A\_U03,IT1A\_U07}
\textbf{W jaki sposób można obliczyć długość tekstu przekazanego jako argument w~poniższej funkcji?}
\begin{lstlisting}[language=c]
void foo(const char* txt) {
. . .
}
\end{lstlisting}
\subsection{Odpowiedź}

\begin{itemize}
\item Używając funkcji strlen z biblioteki string.h
\begin{lstlisting}[language=c]
/* Declaration */
size_t strlen(const char *str)

/* Example */
strlen(txt)
\end{lstlisting}

\item Zliczając ile znaków występuje w tekście od znaku na który wskazuje wskaźnik do znaku końca łańcucha znaków ('\textbackslash0')
\begin{lstlisting}[language=c]
int length = 0;
char c = *txt;
while(c != '\0') {
   length++;
   c = *(++txt);
}
\end{lstlisting}
\end{itemize}
  
\subsection{Wprowadzenie teoretyczne}
Łańuch znaków w języku C to tablica znaków której ostatni element ma wartość \textbackslash0 - null\\
Poniższe deklaracje łańcucha znaków są równoważne:
\begin{lstlisting}[language=c]
char greeting[]  = "Hello";
char greeting[6] = "Hello";
char *greeting   = "Hello";
char greeting[]  = { 'H', 'e', 'l', 'l' 'o', '\0'};
char greeting[6] = { 'H', 'e', 'l', 'l' 'o', '\0'};
\end{lstlisting}

\begin{center}
\includegraphics[width=13cm]{01/string-in-c}
\captionof{figure}{Łańcuchy znaków w C}
\end{center}

Istnieje biblioteka string.h która zawiera funkcja ułatwiające manipulowanie łańcuchami znaków.
\begin{lstlisting}[language=c]
strcpy(s1, s2); 
/*Copies string s2 into string s1*/

strcat(s1, s2);
/*Concatenates string s2 onto the end of string s1*/

strlen(s1);
/*Returns the length of string s1*/

strcmp(s1, s2);
/*Returns 0 if s1 and s2 are the same;
  less than 0 if s1<s2;
  greater than 0 if s1>s2*/
\end{lstlisting}

% 2 ---------------------------------------------------------------------------------------------------------------

\section{IT1A\_W03,IT1A\_U03,IT1A\_U07}
\textbf{Co mozesz powiedzieć o poniższej deklaracji?}
\begin{lstlisting}[language=c]
int t[10]={1, 2, [4]=1}
\end{lstlisting}

\subsection{Wprowadzenie teoretyczne}
W standardzie C99 dodane zostało takie cos jak Designated Initializers. Pozwalaja one inicjalizowac elementy tablicy w dowolnej kolejnosci podając je podczas inicjalizacji w sposób \textbf{\textit{[index]=wartosc}} lub \textbf{\textit{[poczatek ... koniec]=wartosc}}. Jeśli po inicjalizacji elementu w jeden z wymienionych powyżej sposobów pojawi się normalna inicjalizacja np {[2]=5, 4} to indeksem liczby 4 będzie 3 ponieważ przyjmie ona następny wolny indeks w tablicy, tutaj było to 3.

\begin{lstlisting}[language=c]
int tab[]={1,2,3,[4]=4,5}  /* stworzy tablice: 1,2,3,0,4,5 */
int tab[]={[1 ... 2]=5, 1} /* stworzy tablice: 0,5,5,1 */
\end{lstlisting}
	
A co jeśli zrobimy tak:
\begin{lstlisting}[language=c]
int tab[]={1,2,3,[4]=4,{5, 6, 7, 8}}
\end{lstlisting}

\textbf{\textit{Odpowiedź}}: Kod dalej sie poprawnie skompiluje a tablica bedzie miala zawartosc 1,2,3,0,4,5 (zostala wzięta 1 wartosc z bloku {5,6,7,8}) wygenerowany zostanie jedynie warning.\\
	
PS: W konstrukcji: [poczatek ... koniec]=wartosc ważna jest spacja po "poczatek"  $\ddot\smile$\\
	
Źródło: \href{https://gcc.gnu.org/onlinedocs/gcc/Designated-Inits.html}{https://gcc.gnu.org/onlinedocs/gcc/Designated-Inits.html}

\subsection{Kod testowy}
\begin{lstlisting}[language=c]
int tab[]={1,2,3,[4]=4,{5, 6, 7, 8}};

int i=0;
for(i=0; i<10; i++) {
	printf("tab[%d]=%d\n", i, tab[i]);
}
	
return 0;
\end{lstlisting}
% 3 ---------------------------------------------------------------------------------------------------------------

\section{IT1A\_W03,IT1A\_U03,IT1A\_U07} 
\textbf{W jaki sposób obliczyć długość tablicy w funkcji foo()?}
\begin{lstlisting}[language=c]
void foo(double t[]){
   // dlugosc tablicy t?
}
\end{lstlisting}

\subsection{Odpowiedź}
\begin{itemize}
\item W tym wypadku nie jest możliwe obliczenie długości tablicy.\\
\end{itemize}

\subsection{Wprowadzenie teoretyczne}
Nie jest możliwe obliczenie długości tablicy posiadając jedynie wskaźnik do niej.\\
Kompilator języka C potrafi obliczyć długość tablicy jeżeli została ona zadeklarowana w tej samej metodzie w której chcemy sprawdzić długość tablicy:
\begin{lstlisting}[language=c]
int array[] = { 1, 2, 3, 4 };
int n = sizeof(array) / sizeof(array[0]);
printf("%d", n); /* it should returns 4 */

foo(array);
/*...*/
void foo(int t[]){
   int n = sizeof(t) / sizeof(t[0]);
   printf("%d", n); /* it shouldn't returns 4 */
}
\end{lstlisting}

% 4 ---------------------------------------------------------------------------------------------------------------

\section{IT1A\_W03,IT1A\_U03,IT1A\_U07}
\textbf{Która z implementacji funkcji zwracajacej tablice jest poprawna?}

\subsection{Wprowadzenie teoretyczne}

Pamięć w C możemy alokować jedną z 3 funkcji:
\begin{lstlisting}[language=c]
void* malloc(size_t size) /* Alokuje podana 
	ilosc pamieci i nie inicjalizuje jej */
void* calloc(size_t numOfElems, size_t sizeOfElem) /* Dziala podobnie jak 
	malloc z tym, ze inicjalizuje pamiec zerami */
void* realloc(void* memory, size_t newSize) /* Realokuje pamiec moze ja 
	zwiekszyc/zmniejszyc jesli zamiast parametru memory podamy NULL
	dziala jak malloc */
\end{lstlisting}

Pamięć zwalniamy funkcją:
\begin{lstlisting}[language=c]
void free(void* memory);
\end{lstlisting}

Bonus: Dodatkowo alokacje tablicy 2D.
	
\subsection{Przykładowy kod i odpowiedzi}
	
\begin{lstlisting}[language=c]
int *tab1=getTableRealloc(10);
int *tab2;
allocTable(&tab2, 10);
int **tab2d=get2dTable(3,3);

int i=0;
for(i=0; i<10; i++) {
	tab1[i]=i;
	tab2[i]=i;
	printf("tab1[%d]=%d,tab2[%d]=%d\n", i, tab1[i], i, tab2[i]);
}
	
free(tab1);
free(tab2);
free2d(tab2d, 3);

return 0;
}

int* getTable(int n) {
	return (int*)malloc(n*sizeof(int));
}

int* getTableCalloc(int n) {
	return (int*)calloc(n, sizeof(n));
}

int* getTableRealloc(int n) {
	return (int*)realloc(NULL, n*sizeof(int));
}

void allocTable(int **tab, int n) {
	*tab=(int*)malloc(n*sizeof(int));
}

int ** get2dTable(int row, int col) {
	int **tab=(int**)malloc(row*sizeof(int));
	int i=0;
	for(i=0; i<row; i++) tab[i]=(int*)malloc(col*sizeof(int));

	return tab;
}

void free2d(int **tab, int row) {
	int i=0;
	for(i=0; i<row; i++) free(tab[i]);
	free(tab);
}
\end{lstlisting}
% 5 ---------------------------------------------------------------------------------------------------------------

\section{IT1A\_W03,IT1A\_U03,IT1A\_U07}
\textbf{Zakładając, ze wielkość typu char to jeden bajt, short to dwa bajty, a double to osiem bajtów, jaka jest wartość wyrażenia sizeof(x), gdzie x jest zmienną poniższego typu strukturalnego, dla standardowych ustawień kompilatora 32-bitowego?}
\begin{lstlisting}[language=c]
struct {
   char c;
   short i;
   double d;
} x;
\end{lstlisting}

\subsection{Wstęp teoretyczny}
W języku C każdy typ, oprócz char jest dopasowywany odpowiednio w przestrzeni adresowej.
\begin{itemize}
\item char może mieć początek na każdym bajcie w przestrzeni adresowej
\item short musi mieć początek na adresie parzystym
\item int lub float musi mieć początek na adresie podzielnym przez 4
\item long musi mieć początek na adresie podzielnym przez 8
\end{itemize}
\textbf{Przykład ułożenia zmiennych w pamięci: }
\begin{lstlisting}[language=c]
char *p;
char c;
int x;
\end{lstlisting}
Przestrzeń dla p zajmuje 4 bajty, zaraz po nim mamy chara 1 bajtowego. Ale rozmieszczenie 4-bajtowego x wymaga dodania luki:

\begin{lstlisting}[language=c]
char *p;      /* 4 or 8 bytes */
char c;       /* 1 byte */
char pad[3];  /* 3 bytes */
int x;        /* 4 bytes */
\end{lstlisting}
pad[3] reprezentuje fakt, że int musi zacząć się na adresie wielokrotności 4. \\\\
\textbf{Pierwszy przykład dla struktur (maszyna 64-bitowa)}
\begin{lstlisting}[language=c]
struct foo1 {
   char *p;     /* 8 bytes */
   char c;      /* 1 byte
   char pad[7]; /* 7 bytes */
   long x;      /* 8 bytes */
};
\end{lstlisting}
\textbf{Drugi przykład dla struktur}
\begin{lstlisting}[language=c]
struct MixedData      /* After compilation in 32-bit x86 machine */
{
    char Data1;       /* 1 byte */
    char Padding1[1]; /* 1 byte for the following 'short'
	                to be aligned on a 2 byte boundary assuming that
	                the address where structure begins is an even number */
    short Data2;      /* 2 bytes */
    int Data3;        /* 4 bytes - largest structure member */
    char Data4;       /* 1 byte */
    char Padding2[3]; /* 3 bytes to make total size of the structure 12 bytes */
};
\end{lstlisting}
Wydawać by się mogło, że sizeof(MixedData) = 9. Jednak nie - rozmiar struktury wynosi 12 bajtów! Do ostatniego elementu struktury dodany jest padding, dzięki któremu rozmiar struktury będzie wielokrotnością największego elementu struktury (tutaj int - 4 bajty).

\subsection{Odpowiedź na pytanie} 
sizeof(x) = 1 + padding(1) + 2 + padding(4) + 8 = 16 

% 6 ---------------------------------------------------------------------------------------------------------------

\section{IT1A\_W03,IT1A\_U03,IT1A\_U07}
\textbf{Przeanalizuj ponizsza deklaracje. Jakie wartosci wyrazen, w których wystepuja wskazniki p1 i p2 zostana wydrukowane? (Załóz, ze uzywasz 32-bitowego kompilatora.)}
\begin{lstlisting}[language=c]
int t[10];
int *p1=&t[0];
int *p2=&t[8];
\end{lstlisting}

\subsection{Wstęp teoretyczny}

Tablica to tak naprawdę wskaźnik na blok pamięci, dlatego możemy zrobić tak:
\begin{lstlisting}[language=c]
int tab[10];
int *tab2=tab;
\end{lstlisting}

Notacja t[x] to nic innego niż skrót na *(t+x), jak już wiemy t jest wskaźnikiem, a dodanie liczby do wskaźnika to przesunięcie wskazywanego miejsca o wielkość wskazywanego typu(np. jesli wskazuje na typ int to +1 przesunie go o sizeof(int))\\

Tutaj podam kilka zalezności które są prawdziwe:\\
Dla danych:
\begin{lstlisting}[language=c]
int t[10];
int *tmp=t;
int *p1=&t[0];
int *p2=&t[8];
\end{lstlisting}

Prawdziwe jest:
\begin{lstlisting}[language=c]
t==&t[0]
p1==t
p2==(t+8)
p2==(p1+8)
\end{lstlisting}

itp. itd.\\

p1 jest wskaźnikiem na tablicę a p2 wskaźnikiem na 8 element tablicy.

\subsection{Przykładowy kod}

\begin{lstlisting}[language=c]
int t[10];
int *tmp=t;
int *p1=&t[0];
int *p2=&t[8];

printf("t==&t[0]?:%d\n", t==&t[0]);
printf("p1==t?:%d\n", p1==t);
printf("p2==t+8?:%d\n", p2==(t+8));
printf("p2==p1+8?:%d\n", p2==(p1+8));
\end{lstlisting}

% 7 ---------------------------------------------------------------------------------------------------------------

\section{IT1A\_W03,IT1A\_U03,IT1A\_U07} 
\textbf{Przeanalizuj poniższą deklarację w języku C}
\begin{lstlisting}[language=c]
int (*x)(int, int);
\end{lstlisting}

\subsection{Odpowiedź}
\begin{itemize}
\item Deklaracja wskaźnika na funkcję przyjmującą jako parametr dwa integery\\
\end{itemize}

\subsection{Wprowadzenie teoretyczne}
\textbf{Wskaźniki na funkcje}\\
Każda funkcja tak jak i zmienna ma swoje miejsce w pamięci do którego można przypisać wskaźnik.\\
Wskaźnik na funkcje różni się od wskaźników na zmienne deklaracją:
\begin{lstlisting}[language=c]
typ_zwracanej_wartosci (*nazwa_wskaznika)(typ1 argument1, typ2 argument2, ...);
\end{lstlisting}
Przypisując nazwę funkcji bez nawiasów do wskaźnika automatycznie informujemy kompilator, że chodzi nam o adres funkcji.\\
Wskaźnika używamy tak, jak normalnej funkcji, na którą on wskazuje.\\
Przykład:
\begin{lstlisting}[language=c]
#include <stdio.h>

int suma (int lhs, int rhs){
   return lhs+rhs;
}
 
int main (){
   int (*wsk_suma)(int a, int b);
   wsk_suma = suma;
   printf("4+5=%d\n", wsk_suma(4,5));
   return 0;
}
\end{lstlisting}

% 8 ---------------------------------------------------------------------------------------------------------------
\section{IT1A\_W03,IT1A\_U03,IT1A\_U07} 
\textbf{Które stwierdzenia dotyczace operatorów w jezyku C/C++ sa poprawne:}

\subsection{Wprowadzenie teoretyczne}
Chyba wszycy wiemy jak dziala pre i post inkrementacja, ale dla zasady
zakladamy ze z zawsze jest rowne 0;
\begin{lstlisting}[language=c]
y=++z    /* y=1 z=1 */
y=z++    /* y=0 z=1 */
\end{lstlisting}

Operatory += -= itp.:
\begin{lstlisting}[language=c]
z=0; y=0;
y=z+=2;
y==2 && z==2 /* true */
\end{lstlisting}

Działania jednoargumentowe:
\begin{lstlisting}[language=c]
z=1;
-z==-1  /* true, ale po tej operacji z dalej jest rowne 1 */
+z==1   /* true ten operator tak naprawde nic nie zmienia,
	bo nawet dla z=-1 otrzymamy -1, bo +(-1)==-1 */
\end{lstlisting}

\subsection{Przykładowy kod}

\begin{lstlisting}[language=c]
int z=1;
int y=1;

printf("z==++z?: %d\n", z==++z);
printf("z!=z++?: %d\n", z==z++);
z=0; y=0;
y=z+=2;
printf("z=0 y=0, y=z+=2, y==2 z==2?: %d\n", y==2 && z==2);
z=1;
printf("z=1,-z==-1:%d\n", -z == -1);
printf("-z=%d\n", z);
\end{lstlisting}

% 9 ---------------------------------------------------------------------------------------------------------------
\section{IT1A\_W03,IT1A\_U03,IT1A\_U07} 
\textbf{Które stwierdzenia dotyczące modyfikatora static w języku C/C++ są poprawne}

\begin{itemize}
\item Zmienne lokalne

Modyfikator static sprawia, że obiekt w danej funkcji jest umieszczany w tej samej pamięci, co zmienna globalna i nie jest usuwany wraz z zakończeniem funkcji.
\begin{lstlisting}[language=c]
int licznik() {
  static int a;
  a++;
  return a;
}

int main() {
  printf("%d", licznik());  /* wypisze sie 1 */
  printf("%d", licznik());  /* wypisze sie 2 */ 
}
\end{lstlisting}

Zmienne lokalne statyczne są automatycznie inicjalizowane wartością 0.

\item Pola klasy

Statyczne pole klasy jest tworzone tylko raz, w tym samym obszarze co zmienne globalne i cały czas pamięta swoją ostatnią wartość. Jest tworzone zanim zostanie utworzony pierwszy obiekt danej klasy, zazwyczaj - tworzone są już na samym starcie programu
Pole statyczne klasy pamięta swoją wartość pomiędzy obiektami. Nie ważne z jakiego obiektu się do niego odwołasz, wartość będzie ta sama, wspólna dla wszystkich obiektów.

\item Metody

Metody statyczne - można wywołać bez podawania jakiegokolwiek obiektu. Ma to jednak swoją cenę - metody statyczne mają dostęp jedynie do innych zmiennych statycznych i metod statycznych tej klasy. 

\begin{lstlisting}[language=c]
class przyklad {
  private:
    int a, b;
    static int c;
  public:
    static void metoda() {
      c=10; 
      // b=10;         - blad! 'b' nie jest statyczne! nie ma do niego dostepu
      // this->a = 10; - blad!! w metodach statycznych 'this' nie istnieje!!
    }
};
\end{lstlisting}
\end{itemize}

Statyczne zmienne globalne i statyczne funkcje są całkowicie niewidoczne i niedostępne dla nikogo, poza plikiem, w którym zostały zdefiniowane.

% 10 ---------------------------------------------------------------------------------------------------------------
\section{IT1A\_W03,IT1A\_U03,IT1A\_U07} 
\textbf{Dzieki konwencji wywołania funkcji w jezyku C znanej jako \_\_cdecl mozliwa jest implementacja funkcji o zmiennej liczbie argumentów, jak printf(). Które stwierdzenia charakteryzujace funkcje typu \_\_cdecl sa prawdziwe?}

\subsection{Wstęp teoretyczny}

Patrząc z perspektywy programisty C:
\begin{itemize}
\item Wymagany jest chociaz jeden normalny parametr
\item Musimy znać liczbę parametrów, np. przekazać jako normalny argument.(printf bierze ta informacje ze stringa z formatem)
\item Argumenty moga byc dowolnego typu potem wyciagamy je makrem var\_arg
\end{itemize}

Z punktu widzenia Assemblera:
\begin{itemize}
\item Stos jest czyszczony przez obiekt wywołujący, a nie przez samą funkcję dzięki czemu możemy tworzyć funkcje ze zmienną ilością parametrów ponieważ obiekt wywołujący wie ile parametrów przekazał funkcji, a sama funkcja nie ma takiej informacji.
\item Kod asemblera programu z funkcjami w konwencji \_\_cdecl jest wiekszy niż w przypadku innych konwencji ponieważ przy każdym wywołaniu funkcji na poziomie asemblera musi się znaleźć także wyczyzszczenie stosu z parametrów
\end{itemize}

Przykład konwencja stdcall(domyślna dla Win32):\\
push 5	;wrzucamy argument na stos\\
call potega	;wywołujemy funkcje\\

Przykład konwencja cdecl:\\
push 5	;wrzucamy argument na stos\\
call potega	;wywołujemy funkcje\\
add esp, 4	;sami czyścimy stos\\

Jak widać w przypadku cdecl mamy o jedną instrukcje więcej.\\

Istnieje jeszcze kilka innych konwencji wszystkie różnią się kodem assemblera.\\

Jak wybrać ręcznie konwencję dla funkcji?\\
\begin{lstlisting}[language=c]
int __cdecl suma(int n, ...);
int __stdcall suma(int a, ...);  /* Zapytacie jak to moze dzialac jak napisalem, 
	ze tutaj funkcja sama czysci argumenty :D, 
	Odpowiedz to: kompilator czaruje i generuje funkcje vararg w _cdecl */
\end{lstlisting}

Źródło: \href{http://www.p-programowanie.pl/cpp/konwencje-wywolywania-funkcji/}{http://www.p-programowanie.pl/cpp/konwencje-wywolywania-funkcji/}

\subsection{Przykładowy kod}

\begin{lstlisting}[language=c]
printf("sum=%d\n", sum(4, 1, 2, 3, 4));

int __cdecl sum(int n, ...) {
	va_list ap;
	va_start(ap, n);
	int i=0;
	int sum=0;
	for(i=0; i<n; i++) sum+=va_arg(ap, int);
	va_end(ap);
	
	return sum;
}
\end{lstlisting}

% 11 ---------------------------------------------------------------------------------------------------------------

\section{IT1A\_W03,IT1A\_U03,IT1A\_U07} 
\textbf{W jaki sposób przekazywany jest parametr do będący tablicą do funkcji w języku C?}
\begin{lstlisting}[language=c]
int main(int argc, char* argv[]){
   //...
}
\end{lstlisting}

\subsection{Odpowiedź}
\begin{itemize}
\item Na stos jest przekazywany adres pierwszego elementu tablicy\\
\end{itemize}

\subsection{Wprowadzenie teoretyczne}
Tablice są przekazywane do funkcji zawsze jako wskaźnik.\\
Zwykłe zmienne oraz obiekty(struktury) mogą być przekazywane przez wartość lub przez referencję.
\begin{itemize}
\item Przez wartość - na stos trafia kopia danej zmiennej lub obiektu.\\
Jeżeli wewnątrz struktury przekazywanej przez wartość znajduje się wskaźnik (lub tablica) to zostanie skopiowany wskaźnik. W skutek czego oryginał i kopia będą współdzielić jeden wskaźnik / tablicę. Zmieniając coś w kopii paramteru nie zmieniamy jego pierwowzoru.
\item Przez referencję - na stos trafia kopia adresu przesyłanego paramteru. Zmieniając coś w referencji, zmieniamy oryginał.
\end{itemize}

% 12 ---------------------------------------------------------------------------------------------------------------
\section{IT1A\_W03,IT1A\_U03,IT1A\_U07} 
\textbf{Które stwierdzenia odnoszace sie do przydziału pamieci dla zmiennych w jezykach C i C++ sa prawdziwe?}

\subsection{Wprowadzenie teoretyczne}

\begin{itemize}
\item Zmienne statyczne, automatyczne (zmienne globalne, statyczne, lokalne, niealokowane) przechowywane są na stosie
\item Zmienne dynamiczne (alokowane poprzez malloc, new, itp.) przechowywane są na stercie
\item Sterta jest wspólna dla wszytkich programów
\item Stos jest ograniczony co do rozmiaru, co za tym idzie nie można na stosie zaalokowac dużej ilości pamięci należy to zrobić dynamicznie na stercie (to jest ta jedna z mistycznych sytuacji kiedy malloc zwraca null'a)
\end{itemize}

% 13 ---------------------------------------------------------------------------------------------------------------

\section{IT1A\_W04,IT1A\_U04,IT1A\_U21} 
\textbf{Które ze stwierdzeń, odnoszących się do referencji w języku C++ są poprawne?}

\subsection{Wprowadzenie teoretyczne}
Referencja w swym działaniu przypomina wskaźniki. Różnica polega jednak na tym, że do referencji można przypisać adres tylko raz ( w momencie deklaracji ), a jej dalsze używanie niczym się nie różni od używania zwykłej zmiennej. Operacje jakie wykona się na zmiennej referencyjnej, zostaną odzwierciedlone na zmiennej zwykłej, z której pobrano adres.\\
Przykład użycia:
\begin{lstlisting}[language=c]
int i = 0;
int &ref_i = i;
cout << i;      // wypisze 0
ref_i = 1;
cout << i;      // wypisze 1
cout << ref_i;  // wypisze 1
\end{lstlisting}
Różnice między wskaźnikami i referencjami:
\begin{lstlisting}[language=c]
int a,b;
int *wsk_a = &a, *wsk_b = &b;
int &ref_a = a, &ref_b = b;
int &ref_c; // kompilator nie zezwoli na to - referencja niezainicjalizowana

wsk_b = &a; // ok
ref_b = &a; // tak sie nie da
\end{lstlisting}
W C++ można było przekazywać parametry do funkcji przez referencję w podobny sposób jak w C. W C++ jednak używa się referencji w o wiele prostszy sposób ( tak jak zwykły parametr ).

\begin{lstlisting}[language=c]
void zwieksz_c (int *i){
   ++(*i); // ta funkcja jest napisana w stylu C
}
void zwieksz_cpp (int& i){
   ++i;    // ta funkcja wykorzystuje mozliwosci C++
}

// wywolanie funkcji niczym sie nie rozni.
int i = 10;
zwieksz_c(&i);
zwieksz_cpp(&i);
\end{lstlisting}

% 14 ---------------------------------------------------------------------------------------------------------------
\section{IT1A\_W04,IT1A\_U04,IT1A\_U07,IT1A\_U21} 
\textbf{Jezeli podczas wykonania instrukcji w C++:}
\begin{lstlisting}[language=c]
A* ptr = new A();
\end{lstlisting}
\textbf{wygenerowany został wyjatek, jego przyczyna moze byc nastepujaca:}

\subsection{Odpowiedź}
Przychodzą mi do głowy 2 możliwości:
\begin{itemize}
\item W konstuktorze wyrzucany jest wyjątek
\item Brakuje pamięci wtedy mechanizm C++ rzuca wyjątkiem std::bad\_alloc
\end{itemize}

\subsection{Przykładowy kod}
\begin{lstlisting}[language=c]
class Exception {
	
};

class A {
	private:
		int i;
	public:
		A(bool a=false) {
			if(a) throw Exception();
		}
		
		~A() {
			cout << "Descruting-A" << endl;
		}
		
		int getI(void) {
			return i;
		}
		void setI(int i) {
			this->i=i;
		}
};

int main(int argc, char** argv) {
	A *a=NULL;
	try {
		A b;
		a=new A(true);
	} catch(Exception& e) {
		cout << "Exception" << endl;
		if(a==NULL) cout << "a-is-null" << endl;
	}
//	if(a==NULL) cout << "a-is-null";
//	a->setI(10);
//	cout << "Wartosc: " << a->getI() << endl;

	delete a;
	
	return 0;
}

\end{lstlisting}

% 15 ---------------------------------------------------------------------------------------------------------------
\section{IT1A\_W04,IT1A\_U04,IT1A\_U07,IT1A\_U21} 
\textbf{Przeanalizuj fragment kodu w języku C++, w którym pojawia się wywołanie operatora <<:}
\begin{lstlisting}[language=c]
A a;
std::cout<<a;
\end{lstlisting}
Która z podanych implementacji operatora << jest poprawna (przykładowy kod zostanie skompilowany i wykonany)?

\subsection{Wstęp teoretyczny}
Aby przeładować operator << musimy zdefiniować funkcję o prototypie: 
\begin{lstlisting}[language=c]
ostream& operator<<(ostream&, const Klasa&);
\end{lstlisting}
Jeśli przy wypisywaniu informacji o obiekcie potrzebny jest dostęp do prywatnych lub chronionych składowych, to funkcję tę należy też zadeklarować jako zaprzyjaźnioną wewnątrz klasy:
\begin{lstlisting}[language=c]
class Klasa {
  public:
   friend ostream& operator<<(ostream&, const Klasa&);
};
\end{lstlisting}
Przykład implementacji funkcji:
\begin{lstlisting}[language=c]
ostream& operator<<(ostream& str, const Klasa& k) {
   return str << k.nazwa << " (" << k.nazwa2 << ")";
}
\end{lstlisting}
\subsection{Odpowiedź}
Według odpowiedzi z egzaminu 2011, poprawne były odpowiedzi dwie, w których operator był w osobnej funkcji

% 16 ---------------------------------------------------------------------------------------------------------------
\section{IT1A\_W04,IT1A\_U04,IT1A\_U07,IT1A\_U21} 
\textbf{Zdefiniowano szablon (wzorzec) funkcji}
\begin{lstlisting}[language=c]
template<class T>
T suma( T* table, int size)
{
	T t=T();
	for(int i=0; i<size; i++) t+=table[i];
	return t;
}
\end{lstlisting}
\textbf{Proces instancjacji szablonu polega na zastapieniu typów i zmiennych bedacych parametrami szablonu konkretnymi typami i wartosciami, a nastepnie generacji kodu wynikowego. Jakie załozenia musi spełniac typ T, aby instancjacja szablonu była mozliwa?}
\\
Zeby mozna było wywołac szablon funkcji parametrem może być:
\begin{itemize}
\item typ prosty
\item klasa która posiada domyślny konstruktor oraz przeciążony oprator+=
\end{itemize}

\begin{lstlisting}[language=c]
class A {
	
	private:
		int value;
	
	public:
		
		A(int value=0) {
			this->value=value;
		}
		
		int getValue(void) {
			return value;
		}
		
		void setValue(int value) {
			this->value=value;
		}
		
		friend ostream& operator<<(ostream& out, A const& obj) {
			out << obj.value;
		}
		
		A operator+(A const& obj) {
			return A(value+obj.value);
		}
		
		A& operator+=(A const& obj) {
			value+=obj.value;
		}
};

template<class T>
T sum(T arg1, T arg2) {
	return arg1+arg2;
}

template<class T>
T suma(T *table, int size) {
	T t=T();
	for(int i=0; i<size;i++) t+=table[i];
	return t;
}

int main(int argc, char** argv) {
	A a(5),b(3);
	A tab[]={A(1), A(2), A(3), A(4)};
	
	
	cout << "Suma tablicy: " << suma(tab, 4) << endl;
	
	cout << sum(a, b) << endl;
	cout << a << "+" << b << endl;
	
	cout << sum(1, 2) << " " << sum(1.0, 1.1) << endl;
	
	return 0;
}
\end{lstlisting}

% 17 ---------------------------------------------------------------------------------------------------------------
\section{IT1A\_W04,IT1A\_U04,IT1A\_U07,IT1A\_U21} 
\textbf{Klasa B przechowuje wskaźniki do obiektów klasy A w kontenerze vector standardowej biblioteki C++ (STL)}
\begin{lstlisting}[language=c]
class A{...};
class B{
   public:
   std::vector<A*> v;
   void add(A &a) { v.push_back(new A(a));}
   ~B();
}
\end{lstlisting}
\textbf{Która z implementacji destruktora jest poprawna ( kompiluje się, nie prowadzi do błędów wykonania lub wycieków pamięci)?}

\subsection{Odpowiedź}
\begin{itemize}
\item 
\begin{lstlisting}[language=c]
 B::~B(){
    for ( std::vector<A*>::iterator i = v.begin(); i != v.end(); ++i )
       delete *i; // wywolujemy delete na obiekcie, nie na wskazniku do niego!
 }
\end{lstlisting}
\end{itemize}

\subsection{Wprowadzenie teoretyczne}
Dla każdego elementu w wektorze powinien zostać, wywołany operator delete. W szczególności używanie funkcji typu std::erase czy std::remove (i ich wariantów, typu std::remove\_if), czyszczenie pojemnika, itp. same nie dają poprawnego rozwiązania – na elementach musi byd jawnie wywołany operator delete, bo żadna z takich funkcji tego sama nie robi. Jeżeli mamy do czynienia z kontenerem obiektów ( czyli nie przechowujemy danych prymitywnych np. char lub wskaźników ) to możemy wywołać funkcję std::vector<T>::clear() która wywoła na każdym obiekcie operację delete.

% 18 ---------------------------------------------------------------------------------------------------------------

\section{IT1A\_W04,IT1A\_U04,IT1A\_U07,IT1A\_U21}
\textbf{Szablon set<T> zdefiniowany w standardowej bibliotece C++ (STL) przechowuje elementy w drzewiastych strukturach danych. Który z przedstawionych typów danych moze byc zastosowany jako parametr instancjacji szablonu set<T>? }

\subsection{Odpowiedź}
Każdy typ, który ma przeładowany operator < lub jeśli podamy przy tworzeniu kolekcji własny komperator
\textbf{set<Typ, Komperator> kolekcja;}

Komperator ma postać:
\begin{lstlisting}[language=c]
struct CustomComperator {
	bool operator()(A const& lhs, A const& rhs) const {
		return lhs.getValue()<rhs.getValue();
	}
};
\end{lstlisting}

\subsection{Przykładowy kod}
\begin{lstlisting}[language=c]
class A {
	
	private:
		int value;
	
	public:
		
		A(int value=0) {
			this->value=value;
		}
		
		int getValue(void) const {
			return value;
		}
		
		void setValue(int value) {
			this->value=value;
		}
		
		friend ostream& operator<<(ostream& out, A const& obj) {
			out << obj.value;
		}
		
		A operator+(A const& obj) const {
			return A(value+obj.value);
		}
		
		A& operator+=(A const& obj) {
			value+=obj.value;
		}
		
		bool operator<(A const& obj) const {
			return value<obj.value;
		}
};

struct CustomComperator {
	bool operator()(A const& lhs, A const& rhs) const {
		return lhs.getValue()<rhs.getValue();
	}
};

int main(int argc, char** argv) {
	
//	set<A, CustomComperator> mapa;
	set<A> mapa;
	
	mapa.insert(A(4));
	mapa.insert(A(2));
	mapa.insert(A(1));
	
	for(set<A>::iterator it=mapa.begin(); it!=mapa.end(); ++it) {
		cout << *it << endl;
	}
	
	return 0;
}
\end{lstlisting}

% 19 ---------------------------------------------------------------------------------------------------------------

\section{IT1A\_W04,IT1A\_U04,IT1A\_U07,IT1A\_U21}
\textbf{Które ze stwierdzeń odnoszących się do konstruktorów kopiujących i operatorów przypisania w języku C++ są poprawne?}

\subsection{Wprowadzenie teoretyczne}
\begin{itemize}
\item Publiczny konstruktor kopiujący oraz operator przypisania zostaną automatycznie wygenerowane przez kompilator, jeśli programista je pominie.
\item Zarówno automatycznie wygenerowany konstruktor kopiujący, jak i operator przypisania kopiują obiekty bit po bicie.
\item Inicjalizacja (tworzenie obiektu na zasadzie: X x = y;) wywołuje konstruktor kopiujący, nie operator przypisania.
\item Oprócz inicjalizacji, konstruktor kopiujący wywoływany jest gdy przekazuje się lub zwraca obiekty przez wartośd (trzeba zrobid kopię obiektu).
\item W niektórych przypadkach, jeśli kompilator jest sprytny i zrobi np. optymalizację wartości zwracanej, to konstruktor kopiujący może nie zostać wywołany przy wywołaniu funkcji zwracającej obiekt przez wartośd.
\end{itemize}

% 20 ---------------------------------------------------------------------------------------------------------------

\section{IT1A\_W04,IT1A\_U04,IT1A\_U07,IT1A\_U21}
\textbf{Implementacja przeciazonych operatorów C++ powinna odzwierciedlac semantyke operacji na typach wbudowanych. Biorac pod uwage to wymaganie, które z implementacji operatorów dla klasy X zadeklarowanej ponizej jest poprawna? }
\begin{lstlisting}[language=c]
class X
{
friend X&operator+=(X&a, const X&b);
	int x;
public:
	X(int _x=0):x(_x) {}
	X&operator+(const X&o);
	X&operator++(int);
	X&operator-=(const X&o);
};
\end{lstlisting}

\subsection{Wprowadzenie teoretyczne}

Operatory można przeciążać na 2 sposoby:
\begin{itemize}
	\item definiując je jako metodę klasy
	\item definiując je jako funkcję zewnętrzną poza klasą
\end{itemize}
	
Dla przykładu przeładowanie operatora "+":\\
W klasie:\\
- zwracany\_typ operator+(const typ\_\&) const;\\
Poza klasą:\\
- zwracany\_typ operator+(const typ1\&, const typ2\&);\\

Dla operatorów 1 argumentowych na przykładzie ++:\\
W klasie:\\
- zwracany\_typ\& operator++();\\
Poza klasą:\\
- zwracany\_typ\& operator++(typ\&);\\

Teraz pewnie zadajecie sobie pytanie dobra no, ale jak to jest z tymi const. \\
Jest to dość intuicyjne i trzeba się trzymać kilku zasad:
\begin{itemize}
	\item jeśli NIE zmieniamy jakiegoś parametru to dajemy przy nim const
	\item jeśli przeładowujemy w klasie i operator nie zmienia klasy tylko zwraca nowy obiekt to na metodzie dajemy na koncu const
\end{itemize}

Dodatkowe pseudo reguły:
\begin{itemize}
	\item zazwyczaj wszytko przekazujemy jako referencję(\&) chyba że zwracamy nowy obiekt tak jak jest to w przypadku np. operatora +
	\item operator można zdefiniować jako funkcje zaprzyjaźnioną żeby mieć dostęp z zewnątrz do prywatnych pól klasy trzeba tak zrobic np. w przypadku operatora wypisania, np.:
		\begin{lstlisting}[language=c]
	friend ostream& operator<< (ostream&,Foo const&);
		\end{lstlisting}
	\end{itemize}
	
Żywy przyklad mozna zobaczyc tutaj na klasie A.\\

Co do przykladu z pytan:\\
\begin{lstlisting}[language=c]
class X
{
	friend X& operator+=(X &a, const X& b); /* Dobrze */
	int x;
	public:
		X(int _x = 0) : x(_x) { }
		X& operator+(const X&o);  /* Zle cala metoda powinna byc const */
		X& operator++(int);       /* Dobrze, int i tak bedzie kopia w funkcji */
		X& operator-=(const X&o); /* Dobrze */
};
\end{lstlisting}

\subsection{Przykładowy kod}
\begin{lstlisting}[language=c]
class A {
	
	private:
		int value;
	
	public:
		
		A(int value=0) {
			this->value=value;
		}
		
		int getValue(void) const {
			return value;
		}
		
		void setValue(int value) {
			this->value=value;
		}
		
		friend ostream& operator<<(ostream& out, A const& obj) {
			out << obj.value;
		}
		
		A operator+(A const& obj) const {
			return A(value+obj.value);
		}
		
		A& operator+=(A const& obj) {
			value+=obj.value;
		}
		
		bool operator<(A const& obj) const {
			return value<obj.value;
		}
};

int main(int argc, char** argv) {
	
	cout << A(1) << endl;
	
	return 0;
}
\end{lstlisting}

% 21 ---------------------------------------------------------------------------------------------------------------

\section{IT1A\_W04,IT1A\_U04,IT1A\_U07,IT1A\_U21}
\textbf{W języku C++ dostep do informacji o typie obiektu w trakcie wykonania programu umożliwają następujące operatory:}

\subsection{Odpowiedź}
\begin{itemize}
\item typeid
\end{itemize}

\subsection{Wprowadzenie teoretyczne}
typeid jest funkcją która zwraca obiekt klasy std::type\_info lub wyrzucającą obiekt klasy std::bad\_typeid w przypadku gdy używamy tej funkcji z nullową wartością.\\
Przykład:
\begin{lstlisting}[language=c]
#include <iostream>
#include <typeinfo>  //for 'typeid' to work

class Person {
public:
   // ... Person members ...
   virtual ~Person() {}
};

class Employee : public Person {
   // ... Employee members ...
};

int main () {
   Person person;
   Employee employee;
   Person *ptr = &employee;
   // The string returned by typeid::name is implementation-defined
   
   std::cout << typeid(person).name() << std::endl;   // Person
   std::cout << typeid(employee).name() << std::endl; // Employee
   std::cout << typeid(ptr).name() << std::endl;      // Person * 
   std::cout << typeid(*ptr).name() << std::endl;     // Employee
}
\end{lstlisting}

% 22 ---------------------------------------------------------------------------------------------------------------

\section{IT1A\_W04,IT1A\_U04,IT1A\_U07,IT1A\_U21}
\textbf{Zadeklarowano dwie klasy w nastepujacy sposób:}
\begin{lstlisting}[language=c]
class A{
public:
	virtual void f(){printf("VA ");}
	void g(){printf("A ");}
};

class B: public A{
public:
	void f(){printf("VB ");}
	void g(){printf("B ");}
};
\end{lstlisting}
\textbf{oraz utworzono dwa obiekty:}
\begin{lstlisting}[language=c]
A* a1 = new A();
A* a2 = new B();
\end{lstlisting}

\subsection{Wprowadzenie teoretyczne}
Wywołanie metody f na obu obiektach spowoduje wywołanie odpowiednich implementacji danej metody, poniewaz jest ona oznaczona jako wirtualna,
czyli wypisze się "VA VB".\\

Natomiast wywołanie metody g spowoduje wywołanie jedynie implementacji wypisującej "A ", ponieważ metoda nie jest wirtualna, a typ wskaźnika to A, czyli wywołanie:
\begin{lstlisting}[language=c]
a1->g();
a2->g();
\end{lstlisting}
wypisze "B B".

\textbf{Uwagi:}
\begin{itemize}
\item metody wirtualne wywoływane w konstruktorze zachowują się jak niewirtualne, czyli wywołają implementacje metody z danej klasy
\item konstruktor jest tak jakby zawsze wirtualny
\item destruktor powinien być ZAWSZE wirtualny. DLa przykłądu jeśli mamy takie dziedziczenie B dziedziczy po A to jeśli destruktor w klasie A będzie nie wierualny to nie wywoła sie destruktor klasy B - czytaj: bardzo źle :).
\end{itemize}

\subsection{Przykładowy kod}
\begin{lstlisting}[language=c]
class A {
	public:
		virtual void f() {cout << "VA ";}
		void g() {cout << "A ";}
};

class B : public A {
	public:
		void f() {cout << "VB ";}
		void g() {cout << "B ";}
};

class C : public B {
	public:
		void f() { cout << "VC "; }
};

int main(int argc, char** argv) {
	A* a1=new A();
	A* a2=new B();
	B* a3=new C();
	
	a1->f();
	a2->f();
	a3->f();
	
	cout << endl;
	
	a1->g();
	a2->g();
	
	delete a1;
	delete a2;
	
	return 0;
}
\end{lstlisting}